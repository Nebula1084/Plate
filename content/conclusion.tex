\section{总结}
一个完整的车牌字符检测系统主要由上文中论述的车牌定位、字符分割与字符识别三大部分组成。其中,用于车牌定位的方法主要分为两大类,机器学习中的Adaboost方法结合Haar等特征,或者是传统的视觉方法。字符分割任务中,目前普遍采用的是传统方法,笔者曾经尝试过自己实现一个字符分割器,发现特定的方法对于输入图片的要求比较高,对于一些畸形的情况,分割情况很糟糕。字符识别方法也主要分为两大类,一大类机器学习中的各种统计、混合分类器与神经网络分类器方法,另一大类是模式、模板匹配方法,但模式、模板匹配方法提出时间较早,较为不灵活。

另一方面,笔者考察了两个目前较大的开源车牌检测系统,EasyPR与openALPR,这些可以投入使用的系统还是偏向于传统方法,只在字符识别中,采用了目前已经相当成熟的SVM分类器。开发者在回答使用者提出的Issue时,也提到了学术界提出的很多方法,在实际测试中并不能取得很好的结果,原因可能在于实际中输入的图片各种各样,光线、偏斜、扭曲、清晰度等等方面的因素都会影响检测的准确率。

虽然开源项目偏向于传统方法,笔者认为,在未来,随着数据集的增大,机器学习方法将在车牌检测系统中获得更大的应用,这些自适应的方法应对各国各地区格式不一的车牌、应对实际拍摄中的偏差应该更为有优势。事实上,车牌检测是一个古老的任务,但是直到今天,其准确率还是不能令人满意的,说明传统方法难以越过某个准确度上界,而还在发展中的机器学习方法应用于实际问题时还需要更多方面的考量。笔者认为,随着机器学习与计算机视觉的发展,这个问题应该可以找到更好的解决方案。
